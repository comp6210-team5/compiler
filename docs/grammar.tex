\documentclass{article}
% generic imports and definitions

\usepackage[T1]{fontenc}
\usepackage[margin=1in]{geometry}
\usepackage[parfill]{parskip}

\usepackage{amsmath}
\usepackage{amssymb}
\usepackage{amsfonts}

\pagenumbering{gobble}

% https://ctan.org/pkg/syntax-mdw
% provides grammar environment for CFG typesetting
\usepackage{syntax}
\setlength{\grammarparsep}{2pt}
\def\lrep{\synshortsoff\texttt{\{}\synshorts~}
\def\rrep{\synshortsoff\texttt{\}}\synshorts~}
\def\lopt{\synshortsoff\texttt{[}\synshorts~}
\def\ropt{\synshortsoff\texttt{]}\synshorts~}
\def\lgrp{\synshortsoff\texttt{(}\synshorts~}
\def\rgrp{\synshortsoff\texttt{)}\synshorts~}
\usepackage{hyperref}

\title{The C Programming Language Specification}

%TODO -- add explanations, reorganize sections they kinda everywhere, and make pretty 

\begin{document}
\maketitle

\section*{Notation}
The syntax is specified using a variant of Extended Backus-Naur Form (EBNF) defined in the \href{https://go.dev/ref/spec#Notation}{Go Programming Language Specification}:
\begin{grammar}
  <Syntax> ::= \lrep <Production> \rrep .
  
  <Production> ::= `<' "production_name" `>' `::=' \lopt <Expression> \ropt `.' .
  
  <Expression> ::= <Term> \lrep `|' <Term> \rrep .
  
  <Term> ::= <Factor> \lrep <Factor> \rrep .
  
  <Factor> ::= "production_name" | "token" \lopt `\ldots' "token" \ropt | <Group> | <Option> | <Repetition> .
  
  <Group> ::= `(' <Expression> `)' .
  
  <Option> ::= `[' <Expression> `]' .
  
  <Repetition> ::= `{' <Expression> `}' .

\end{grammar}
% TODO: explain
\section*{Source code representation}
Source code must be plain ASCII text. Additionally, of the ASCII
characters, only the following belonging to these categories may
appear in the source text:
\begin{grammar}
  <newline> ::= "the ASCII code 10 (0x0a)" .

  <white_space> ::= \lgrp <new_line> | "the ASCII code 32 (0x20)" | "the ASCII code 13 (0x0d)" \rgrp \lrep <new_line> | "the ASCII code 32 (0x20)" | "the ASCII code 13 (0x0d)" \rrep .

  <letter> ::= 'A' \ldots 'Z' | 'a' \ldots 'z' | '_' .

  <decimal_digit> ::= '0' \ldots '9' .

  <binary_digit> ::= '0' | '1' .

  <octal_digit> ::= '0' \ldots '7' .

  <hex_digit> ::= '0' \ldots '9' | 'A' \ldots 'F' | 'a' \ldots 'f' .

  <printable_ascii> ::= "the ASCII code 32 (0x20)" \ldots "the ASCII code 126 (0x7e)" .
  
\end{grammar}
Note that this disallows most control characters, the NUL character, and any
extended ASCII characters (above 127). Arbitrary bytes may be represented in
strings using various escape sequences (namely \TT{\textbackslash xXX}. Also
note that we allow only newlines, spaces, and tabs as whitespace.
  
\section*{Lexical elements}
\subsection*{Comments}
Comments are ignored by the compiler and may come in two forms: (1)
single-line comments beginning with
\texttt{\textbackslash\textbackslash} and (2) multi-line comments
beginning with \texttt{\textbackslash*} and terminated with
\texttt{*\textbackslash}.

Note that the compiler identifies and treats these separately from
tokens (i.e. they are ignored rather than tokenized), since they don't
have impact on program meaning/form.

\subsection*{Tokens}
There are four classes of tokens recognized by the compiler: \emph{identifiers},
\emph{keywords}, \emph{operators \& punctuation}, and \emph{literals}. Also,
we further split literals (at the token stage) into \emph{integer
literals}, \emph{floating-point literals}, \emph{string literals}, and
\emph{character literals} to aid parsing. \emph{white\_space} is ignored except
as it separates tokens, in which case one or more ``white space'' characters are
essentially compressed into one white space rule noted in the character
categories above. A token in the source code consists of the longest sequence of
characters which match its definition (i.e. the lexer is greedy).

\subsection*{Identifiers}
% TODO: do we want to allow '$' in identifiers?
\begin{grammar}
  <identifier> ::= <letter> \lrep <letter> | <digit> \rrep
\end{grammar}

\subsection*{Keywords}
Keywords are any of the following (case-insensitive):
% TODO: expand
\begin{verbatim}
if, else, return, switch, case, while, break, continue
\end{verbatim}

\subsection*{Operators \& Punctuation}
The following are recognized as either operators or punctuation:
% TODO: '\', is that for line-breaks?
% TODO: we're definitely still missing some, i.e. << >> %= += -= *= <<= |= and c.
\begin{verbatim}
; : ? + - ~ % / * < > = ! & | [ ] ^ ( ) , { } \ != == <= && || ++ --
\end{verbatim}

\subsection*{Literals}
\subsubsection*{Integer literals}
\begin{grammar}
  <integer_lit> ::= <decimal_lit> | <hex_lit> | <binary_lit> | <octal_lit> .

  <decimal_lit> ::= \lgrp `0' \ldots `9' \rgrp \lopt \lopt `\'' \ropt <decimal_digits> \ropt .

  <hex_lit> ::= `0' \lgrp `x' | `X' \rgrp \lopt `\'' \ropt <hex_digits> .

  <binary_lit> ::= `0' \lgrp `b' | `B' \rgrp \lopt `\'' \ropt <binary_digits> .

  <octal_lit> ::= `0' \lgrp `o' | `O' \rgrp \lopt `\'' \ropt <octal_digits> .

  <decimal_digits> ::= <decimal_digit> \lrep \lopt `\'' \ropt <decimal_digit> \rrep .

  <hex_digits> ::= <hex_digit> \lrep \lopt `\'' \ropt <hex_digit> \rrep .

  <binary_digits> ::= <binary_digit> \lrep \lopt `\'' \ropt <binary_digit> \rrep .

  <octal_digits> ::= <octal_digit> \lrep \lopt `\'' \ropt <octal_digit> \rrep .
\end{grammar}

For readability, a single \texttt{\'} character may appear after a base prefix or between successive digits.

\subsubsection*{Floating-point literals}
\begin{grammar}
  <float_lit> ::= <decimal_float_lit> . %TODO: add hex float literals

  <decimal_float_lit> ::= <decimal_digits> `.' \lopt <decimal_digits> \ropt
  \alt \lopt <decimal_digits> \ropt `.' <decimal_digits> .

  %TODO: add exponentials
\end{grammar}

\subsubsection*{Character literals}
\begin{grammar}
  <char_lit> ::= `\'' \lgrp <char_ascii_value> | <byte_value> \rgrp `\'' .

  <char_ascii_value> ::= <char_ascii> | <escaped_char> .

  % TODO: define an exception syntax in notation
  <char_ascii> ::= <printable_ascii> "\textbf{except}" \lgrp `\textbackslash' | `\'' \rgrp .

  <byte_value> ::= <octal_byte_value> | <hex_byte_value> .

  <octal_byte_value> ::= `\textbackslash' <octal_digit> <octal_digit> <octal_digit> .

  <hex_byte_value> ::= `\textbackslash x' <hex_digit> <hex_digit> .

  % https://en.wikipedia.org/wiki/Escape_sequences_in_C
  % except '?' because I don't know what trigraphs are
  % Also added the actual newline for "multi-line\
  % strings" in source-code. Or multi-line character literals lol
  <escaped_char> ::= `\textbackslash' \lgrp `a' | `b' | `e' | `f' | `n' | `r' |
  `t' | `v' | `\textbackslash' | `\'' | `\"' | "The ASCII char 0x0a (newline)" \rgrp .
\end{grammar}
The actual newline escaped character (as opposed to the representative
\TT{``\textbackslash n''}) is included to allow multi-line strings. In actual C,
the backslash-newlines are kind of ignored in the source-text
entirely,\footnote{Actually deleted; q.v. N1570 Section 5.1.1.2 Note 1 sub 2}
and thus aren't defined at the level of strings. We can't see much point in
including them, outside of strings, aside from say really big numbers.

\subsubsection*{String literals}
\begin{grammar}
  <string_lit> ::= `\"' \lrep <string_ascii_value> | <byte_value> \rrep `\"' .

  <string_ascii_value> ::= <string_ascii> | <escaped_char> .

  <string_ascii> ::= <printable_ascii> "\textbf{except}" \lgrp `\textbackslash' | `\"' \rgrp .
\end{grammar}
% TODO: In the C11 standard, section 6.4.5 (string literals), a sequence of
% adjacent string literals separated only by whitespace is concatenated into a
% single multibyte character sequence during a translation phase, which is a
% cleaner way of supporting multi-line strings anyways.


\section*{Declarations}
\begin{grammar}
  % I think this could be simplified as \lrep `*' \rrep in the <Declarator> rule
  <Pointer> ::= `*' \lopt <Pointer> \ropt .
  
  <Declarator> ::= \lopt <Pointer> \ropt <identifier> | \lopt <Pointer> \ropt
  `(' <Declarator> `)' .
  
  <InitDeclarator> ::= <Declarator> | <Declarator> `=' <Initializer> .
  
  <VarDecl> ::= <TypeSpecifier> <InitDeclarator> \lrep `,' <InitDeclarator>
  \rrep `;' .
  
  <Declaration> ::= <VarDecl> .
  
  <TopLevelDecl> ::= <Declaration> | <FunctionDecl> `;' | <FunctionDefinition> .
\end{grammar}
We (for now) differentiate between a function declaration and function
definition (to avoid including the \TT{`;'} in the function declaration rule),
though a function definition basically includes a function declaration (whether
its been declared previously or not).
% TODO: adding arrays requires solving a left-recursion of the above Declarator
% rule to achieve <Declarator> ::= \lopt <Pointer> \ropt <Declarator> `[' stuff
% `]'
\subsection*{Functions}
\begin{grammar}
  <ParameterList> ::= <TypeSpecifier> <Declarator> \lrep `,' <TypeSpecifier>
  <Declarator> \rrep .

  <FunctionDecl> ::= <TypeSpecifier> <Declarator> `(' \lopt <ParameterList>
  \ropt `)' .

  <FunctionDefinition> ::= <FunctionDecl> <CompoundStatement> .
\end{grammar}

\section*{Types}
A \synshorts <TypeSpecifier> \synshortsoff may be any of the following:
\begin{verbatim}
void char short int long float double
\end{verbatim}
% TODO: add structs, maybe enums?
% TODO: add signed and unsigned, and with that, the ability to combine type
% specifiers (N1570 section 6.7.2 constraints)
% TODO: type names and typedefs

\subsection*{Statements and Blocks}

\begin{grammar}
  <Expression> ::= <AssignmentExpression> \lrep `,' <AssignmentExpression> \rrep .
  
  <ExpressionStmt> ::= \lopt <Expression> \ropt `;' .

  % TODO: add iteration, flow-control, etc.
  <Statement> ::= <ExpressionStmt> .
  
  <CompoundStatement> ::= `\{' \lrep <Declaration> \rrep \lrep <Statement>
  \rrep `\}' .
\end{grammar}

\section*{Expressions}
% We shall try N1570's inheritence-based approach to operator precedence
% and maybe that just works

% TODO: add some necessary constraints like e.g. "you can't array-index an
% integer constant", all of the constraints (some of which are necessary for us)
% are listed in n1570 section 6.5)

\begin{grammar}
  <PrimaryExpression> ::= <identifier> | <literal> | `(' <Expression> `)' .

  % TODO: the C grammar distinguishes an argument expression list in function
  % calls from a normal parenthesized ( <Expression> ) in order to distinguish
  % the "comma operator" as it appears in a normal <Expression> list vs. when it
  % appears in an argument list. Is this distinction important to us? Do we even
  % want to support expressions as a sequentially evaluated list of single
  % expressions? (i.e. the Comma operator, N1570 section 6.5.17)
  <PostfixTail> ::= `[' <Expression> `]' | `(' \lopt <AssignmentExpression>
  \lrep `,' <AssignmentExpression> \rrep \ropt `)' | `++' | `- -' .

  % TODO: add struct dot operator (and maybe pointer dereference, though that's
  % unnecessary) when support for structs is added
  <PostfixExpression> ::= <PrimaryExpression> \lrep <PostfixTail> \rrep .

  % TODO: typesetting `--' merges into one hyphen.
  <UnaryOp> ::= `&' | `*' | `+' | `-' | `~' | `!' | `++' | `- -' .

  <UnaryExpression> ::= <PostfixExpression> | <UnaryOp> <UnaryExpression> .
\end{grammar}
The expression \TT{`+++i'} will be evaluated as \TT{`++(+i)'} due to the
order that our tokenizer recognizes tokens.
% TODO: also I didn't know that `+' is a unary op in C... Does it force the
% expression always positive or treat it as unsigned?

% TODO: casting?

\begin{grammar}
  <MultiplicativeTail> ::= `*' <UnaryExpression> | `/' <UnaryExpression> | `\%'
  <UnaryExpression> .
  
  <MultiplicativeExpression> ::= <UnaryExpression> \lrep <MultiplicativeTail>
  \rrep .

  <AdditiveTail> ::= `+' <MultiplicativeExpression> | `-'
  <MultiplicativeExpression> .

  <AdditiveExpression> ::= <MultiplicativeExpression> \lrep <AdditiveTail> \rrep .

  <ShiftTail> ::= `<<' <AdditiveExpression> | `\>\>' <AdditiveExpression> .
  
  <ShiftExpression> ::= <AdditiveExpression> \lrep <ShiftTail> \rrep .

  <RelationalTail> ::= `<' <ShiftExpression> | `\>' <ShiftExpression> | `<='
  <ShiftExpression> | `\>=' <ShiftExpression> .
  
  <RelationalExpression> ::= <ShiftExpression> \lrep <RelationalTail> \rrep .

  <EqualityTail> ::= `==' <RelationalExpression> | `!=' <RelationalExpression> .
  
  <EqualityExpression> ::= <RelationalExpression> \lrep <EqualityTail> \rrep .

  <AndExpression> ::= <EqualityExpression> \lrep `&' <EqualityExpression> \rrep
  .

  <XorExpression> ::= <AndExpression> \lrep `^' <AndExpression> \rrep .

  <OrExpression> ::= <XorExpression> \lrep `|' <XorExpression> \rrep .

  <LogicalAndExpression> ::= <OrExpression> \lrep `&&' <OrExpression> \rrep .

  <LogicalOrExpression> ::= <LogicalAndExpression> \lrep `||'
  <LogicalAndExpression> \rrep .

  % TODO: conditional (i.e. ternary) expressions?

  <AssignmentOp> ::= `=' | `*=' | `/=' | `\%=' | `+=' | `-=' | `<<=' | `\>\>='
  | `&=' | `^=' | `|=' .
  
  <AssignmentExpression> ::= <LogicalOrExpression> | <UnaryExpression>
  <AssignmentOp> <AssignmentExpression> .
\end{grammar}

\end{document}

% May the graveyard rest in git where it belongs

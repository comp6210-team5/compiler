\documentclass{article}
% generic imports and definitions

\usepackage[T1]{fontenc}
\usepackage[margin=1in]{geometry}
\usepackage[parfill]{parskip}

\usepackage{amsmath}
\usepackage{amssymb}
\usepackage{amsfonts}

\pagenumbering{gobble}

\def\TT[1]{\texttt{#1}}


% https://ctan.org/pkg/syntax-mdw
% provides grammar environment for CFG typesetting
\usepackage{syntax}
\setlength{\grammarparsep}{2pt}
\def\lrep{\synshortsoff\texttt{\{}\synshorts~}
\def\rrep{\synshortsoff\texttt{\}}\synshorts~}
\def\lopt{\synshortsoff\texttt{[}\synshorts~}
\def\ropt{\synshortsoff\texttt{]}\synshorts~}
\def\lgrp{\synshortsoff\texttt{(}\synshorts~}
\def\rgrp{\synshortsoff\texttt{)}\synshorts~}
\usepackage{hyperref}

\title{The C Programming Language Specification}

\begin{document}
\maketitle

\section*{Notation}
The syntax is specified using a variant of Extended Backus-Naur Form (EBNF) defined in the \href{https://go.dev/ref/spec#Notation}{Go Programming Language Specification}:
\begin{grammar}
  <Syntax> ::= \lrep <Production> \rrep .
  
  <Production> ::= `<' "production_name" `>' `::=' \lopt <Expression> \ropt `.' .
  
  <Expression> ::= <Term> \lrep `|' <Term> \rrep .
  
  <Term> ::= <Factor> \lrep <Factor> \rrep .
  
  <Factor> ::= "production_name" | "token" \lopt `\ldots' "token" \ropt | <Group> | <Option> | <Repetition> .
  
  <Group> ::= `(' <Expression> `)' .
  
  <Option> ::= `[' <Expression> `]' .
  
  <Repetition> ::= `{' <Expression> `}' .

  <Test> ::= \lgrp <Expression> \rgrp .
  
\end{grammar}
% TODO: explain
\section*{Source code representation}
Source code must be plain ASCII text. Additionally, of the ASCII
characters, only the following belonging to these categories may
appear in the source text:
\begin{grammar}
  <newline> ::= "the ASCII code 10 (0x0a)" .

  <white_space> ::= \lgrp <new_line> | "the ASCII code 32 (0x20)" | "the ASCII code 13 (0x0d)" \rgrp \lrep <new_line> | "the ASCII code 32 (0x20)" | "the ASCII code 13 (0x0d)" \rrep .

  <letter> ::= 'A' \ldots 'Z' | 'a' \ldots 'z' | '_' .

  <decimal_digit> ::= '0' \ldots '9' .

  <binary_digit> ::= '0' | '1' .

  <octal_digit> ::= '0' \ldots '7' .

  <hex_digit> ::= '0' \ldots '9' | 'A' \ldots 'F' | 'a' \ldots 'f' .

  <printable_ascii> ::= "the ASCII code 32 (0x20)" \ldots "the ASCII code 126 (0x7e)" .
  
\end{grammar}
Note that this disallows most control characters, the NUL character,
and any extended ASCII characters (above 127). Arbitrary bytes may be
represented in strings using various escape sequences. %TODO: described later
\section*{Lexical elements}
\subsection*{Comments}
Comments are ignored by the compiler and may come in two forms: (1)
single-line comments beginning with
\texttt{\textbackslash\textbackslash} and (2) multi-line comments
beginning with \texttt{\textbackslash*} and terminated with
\texttt{*\textbackslash}.

Note that the compiler identifies and treats these separately from
tokens (i.e. they are ignored rather than tokenized), since they don't
have impact on program meaning/form.

\subsection*{Tokens}
There are four classes of tokens recognized by the compiler:
\emph{identifiers}, \emph{keywords}, \emph{operators \& punctuation},
and \emph{literals}. \emph{white\_space} is ignored except as it
separates tokens, in which case one or more ``white space'' characters
are essentially compressed into one white space rule noted in the
character categories above. A token in the source code consists of the
longest sequence of characters which match its definition (i.e. the
lexer is greedy).

\subsection*{Identifiers}
\begin{grammar}
  <identifier> ::= <letter> \lrep <letter> | <digit> \rrep
\end{grammar}

\subsection*{Keywords}
Keywords are any of the following (case-insensitive):
\begin{verbatim}
if, else, return, switch, case, while, break, continue
\end{verbatim}

\subsection*{Operators \& Punctuation}
The following are recognized as either operators or punctuation:
% TODO: is '?' a valid operator in C?
% TODO: and '\', is that for line-breaks?
\begin{verbatim}
;, :, ?, +, -, /, *, <, >, =, !, &, |, [, ], ^, (, ), ,, {, }, \, !=, ==, <=, &&, ||, ++, --
\end{verbatim}

\subsection*{Literals}
\subsubsection*{Integer literals}
\begin{grammar}
  <integer_lit> ::= <decimal_lit> | <hex_lit> | <binary_lit> | <octal_lit> .

  <decimal_lit> ::= \lgrp `0' \ldots `9' \rgrp \lopt \lopt `\'' \ropt <decimal_digits> \ropt .

  <hex_lit> ::= `0' \lgrp `x' | `X' \rgrp \lopt `\'' \ropt <hex_digits> .

  <binary_lit> ::= `0' \lgrp `b' | `B' \rgrp \lopt `\'' \ropt <binary_digits> .

  <octal_lit> ::= `0' \lgrp `o' | `O' \rgrp \lopt `\'' \ropt <octal_digits> .

  <decimal_digits> ::= <decimal_digit> \lrep \lopt `\'' \ropt <decimal_digit> \rrep .

  <hex_digits> ::= <hex_digit> \lrep \lopt `\'' \ropt <hex_digit> \rrep .

  <binary_digits> ::= <binary_digit> \lrep \lopt `\'' \ropt <binary_digit> \rrep .

  <octal_digits> ::= <octal_digit> \lrep \lopt `\'' \ropt <octal_digit> \rrep .
\end{grammar}

For readability, a single \texttt{\'} character may appear after a base prefix or between successive digits.

\subsubsection*{Floating-point literals}
\begin{grammar}
  <float_lit> ::= <decimal_float_lit> . %TODO: add hex float literals

  <decimal_float_lit> ::= <decimal_digits> `.' \lopt <decimal_digits> \ropt
  \alt \lopt <decimal_digits> \ropt `.' <decimal_digits> .

  %TODO: add exponentials
\end{grammar}

\subsubsection*{Character literals}
\begin{grammar}
  <char_lit> ::= `\'' \lgrp <char_ascii_value> | <byte_value> \rgrp `\'' .

  <char_ascii_value> ::= <char_ascii> | <escaped_char> .

  % TODO: define an exception syntax in notation
  <char_ascii> ::= <printable_ascii> "\textbf{except}" \lgrp `\textbackslash' | `\'' \rgrp .

  <byte_value> ::= <octal_byte_value> | <hex_byte_value> .

  <octal_byte_value> ::= `\textbackslash' <octal_digit> <octal_digit> <octal_digit> .

  <hex_byte_value> ::= `\textbackslash x' <hex_digit> <hex_digit> .

  % https://en.wikipedia.org/wiki/Escape_sequences_in_C
  % except '?' because I don't know what trigraphs are
  <escaped_char> ::= `\textbackslash' \lgrp `a' | `b' | `e' | `f' | `n' | `r' | `t' | `v' | `\textbackslash' | `\'' | `\"' \rgrp .
\end{grammar}

\subsubsection*{String literals}
\begin{grammar}
  <string_lit> ::= `\"' \lrep <string_ascii_value> | <byte_value> \rrep `\"' .

  <string_ascii_value> ::= <string_ascii> | <escaped_char> .

  <string_ascii> ::= <printable_ascii> "\textbf{except}" \lgrp `\textbackslash' | `\"' \rgrp .
\end{grammar}



%% additions start here 

\section*{Types}
\begin{grammar}
	<Type> ::= <TypeName> `(' <TypeArgs> `)' | <TypeLit> | `[' <Type> `]' .

	<TypeName> ::= <identifier> . %Left out -- | QualifiedIdent

	<TypeArgs> = `(' <TypeList> \lopt `,' \ropt `)' .

	<TypeList> = <Type> \lrep `,' <Type> \rrep .

	<TypeLit> = <ArrayType> | <StructType> | <FunctionType> | <InterfaceType> %Maybe leave out interfaces? -- Left out PointerType, SliceType, MapType, ChannelType
\end{grammar}

%\subsection*{Boolean types}
	
%\subsection*{Numberic Types} 

\subsection*{Array Types}
\begin{grammar}
	<ArrayType> = `(' ArrayLength `)' ElementType .

	<ArrayLength> = Expression .
 
	<ElementType> = <Type> . 
\end{grammar}


%SliceTypes left out 

\subsection*{Struct Types} 
\begin{grammar} 
	<StructType> = `struct' `\{'  \lrep <FieldDecl> `;'  \rrep `\}' .

	<FieldDecl> = \lgrp<IdentifierList Type> | <EmbeddedField>\rgrp \lopt <Tag> \ropt . 

	<EmbeddedField> = \lopt `*' \ropt <TypeName> \lopt <TypeArgs> \ropt .

	<Tag> = <string_lit> .  
\end{grammar}

%PointerTypes left out 


\subsection*{Function Types} 
\begin{grammar} 
	<FunctionType> = `func' <Signature> .

	<Signature> = <Parameters> \lopt <Result> \ropt .

	<Result> = <Parameters> | <Type> .

	<Parameters> = "(" \lopt <ParameterList> \lopt "," \ropt \ropt ")" .

	<ParameterList> = <ParameterDecl> \lrep "," <ParameterDecl> \rrep .

	<ParameterDecl> = \lopt <IdentifierList> \ropt \lopt "..." \ropt <Type> .
\end{grammar}


%Maybe leave out interfaces? 
\subsection*{Interface Types} 
\begin{grammar}
	<InterfaceType>  = `interface' `\{' \lrep <InterfaceElem> `;' \rrep `\}' .

	<InterfaceElem> = <MethodElem> | <TypeElem> .

	<MethodElem> = <MethodName Signature> .

	<MethodName> = <identifier> .

	<TypeElem> = <TypeTerm> \lrep `|' <TypeTerm> \rrep .

	<TypeTerm> = <Type> | %<UnderlyingType> .

	<UnderlyingType>  = `~' <Type> .
\end{grammar} 


%Map Types left out

%Channel Types left out 

\section*{Blocks} 
\begin{grammar} 
	<Block> = `\{' <StatementList> `\}' . 

	<StatementList> = \lrep <Statement> `;' \rrep .

\end{grammar}

\section*{Declarations and Scope} 
\begin{grammar} 
	<Declaration> = <ConstDecl> | <TypeDecl> | <VarDecl> .

	<TopLevelDecl> = <Declaration> | <FunctionDecl> | <MethodDecl> . 
\end{grammar} 

\subsection*{Constant Declaration}
\begin{grammar}
	<ConstDecl> = `const' \lgrp <ConstSpec> | `(' \lrep <ConstSpec> `;' \rrep `)' \rgrp .

	<ConstSpec> = <IdentifierList> \lrep \lrep <Type> `=' <ExpressionList> \rrep

	<IdentifierList> = <identifier> \lrep `,' <identifier> \rrep . 

	<ExpressionList> = <Expression> \lrep `,' <Expression> \rrep . 
\end{grammar} 

\subsection*{Type Declarations} 
%binds an identifier to a <Type>. 
\begin{grammar} 
	<TypeDecl> = `type' \lgrp <TypeSpec>  | `(' \lrep <TypeSpec> `;' \rrep `)' \rgrp .

	<TypeSpec> = <AliasDecl> | <TypeDef>  .
\end{grammar} 

\subsection*{Alias Declarations}
%Binds identifier to type
\begin{grammar} 
	<AliasDecl> = <identifier> `=' <Type> . 
\end{grammar}



\subsection*{Type Definitions}
%Creates new type with same <UnderlyingType> and operations as given type && binds an identifier
%Not sure if this stays
\begin{grammar} 
	<TypeDef> = <identifier> \lopt <TypeParameters> \ropt  <Type> .
\end{grammar}

\subsection*{Type Parameter Declarations}
%Declares type param for a generic function or type declaration.
%All non-blank names in list must be unique
%Type parameter replaces w/ <TypeArguement> upon instantiation of generic function or type
\begin{grammar} 
	<TypeParameters> = `[' <TypeParamList> \lopt 	`,' \ropt `]' . 

	<TypeParamList> = <TypeParamDecl> \lrep `,' <TypeParamDecl> \rrep .
	
	<TypeParamDecl> = <IdentifierList> <TypeConstraint> . 
\end{grammar} 

\subsection*{Type Constraints}
%interface that defines the set of permissible type arguments for the repective type param && controls the operations supported by values of that type param
%[T ~int] -- okay  
%type Constraint ~int -- illegal: ~int not in type param list
\begin{grammar} 
	<TypeConstraint> = <TypeElem> .
\end{grammar} 

\section{Variable Declarations} 
%Creates one or more <Variables>, binds corresponding identifier, && gives type and initial value
%i.e. var i int
\begin{grammar} 
	<VarDecl> = `var' \lgrp <VarSpec> | `(' \lrep <VarSpec> `;' \rrep `)' \rgrp .

	<VarSpec> = <IdentifierList> \lgrp <Type> \lopt `=' <ExpressionList> \ropt | `=' <ExpressionList> \lgrp .

\end{grammar}

\subsection*{Short variable declarations}
%shorthand for <variable declaration> 
% i, j := 0, 10 
\begin{grammar} 
	<ShortVarDecl> = <IdentifierList> `:=' <ExpressionList> .
\end{grammar}

\section*{Function Declarations} 
%binds identifier to a function
\begin{grammar} 
	<FunctionDecl> = `func' <FunctionName> \lopt <TypeParameters> \ropt <Signature> \lopt <FunctionBody> \ropt .
	
	<FunctionName> = <identifier> .
	
	<FunctionBody> = <Block> .
\end{grammar} 

































\end{document}